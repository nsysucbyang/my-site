\documentstyle[12pt,psfig]{article}
%\documentstyle[12pt,twocolumn,psfig]{article}
\parindent=32truept
\topmargin=-1.2in
\oddsidemargin=0.0truein
\evensidemargin=0.0truein
%default:(5.8, 8.7in)
\textwidth=6.3truein
\textheight=10.in
\begin{document}
\pssilent
\psdraft
%\pagenumbering{roman}

\normalsize

\pagestyle{plain}
%\pagestyle{empty}
\pagenumbering{arabic}
\setcounter{page}{1}
\baselineskip=18pt
\parskip=12pt

\begin{center}
{\LARGE\bf National Collegiate Programming Contest 1996 (Taiwan)}
\end{center}

\begin{center}
{\Large\bf Problem A: Area of an Object}
\end{center}

\bigskip\noindent{\bf The Problem}

Assume that there is an object in a digital picture whose boundary is
specified by a sequence of moves.
A move can only be in the $8$ directions to its neighbours.
The $8$ directions are defined as follows.
\begin{verbatim}
3 2 1
4 . 0
5 6 7
\end{verbatim}
A sequence of moves represented by the numbers 0, 1, 2, 3, 4, 5, 6, 7 are
called a {\it chain code}.
Assume that the object has no holes in it.
An object in a digital picture can then be specified by a starting
point, followed a chain code.
For example, with starting point $(1,5)$, the object shown at the left
of the following figure can be represented by a string at the right.
\begin{verbatim}
0000X0000000X
000XXXXX0000X
XXXXXXXXXXXX0                1,5 554400070071721012654443443
0000XXXXXX000
0000000X0X000
\end{verbatim}
Write a program to calculate the area of the object in a picture
by counting how many {\tt X}'s are covered by the object.

\bigskip\noindent{\bf The Input}

Each object in the picture is represented by a starting point and
followed by a chain code.
The chain code may be more than 80 characters.

\bigskip\noindent {\bf The Output}

If the chain code is a sequence of moves from the starting point back to
the same point, then print out the area of the object represented by
the chain code, otherwise, print the error message ``Illegal chain code.''

\bigskip\noindent {\bf Sample Input}
\begin{verbatim}
1,40 5445456667770100110700071113334432544353*
1,24 5676003124*
\end{verbatim}

\bigskip\noindent {\bf Sample Output}
\begin{verbatim}
object 1: the area is 123
object 2: Illegal chain code.
\end{verbatim}
\vspace{1cm }

\begin{center}
{\Large\bf Problem B: Binary String Combination}
\end{center}

\bigskip\noindent{\bf The Problem}

Consider {\it n-bit} binary string with exact $k$ 1's.
Obviously, there are $n\choose k$ binary strings with this property.
For example, the list of 5-bit binary strings with two 1's is as follows:
\begin {center}
00011, 00101, 00110, 01001, 01010, 01100, 10001, 10010, 10100, 11000
\end{center}
There are totally ${5\choose 2}=10$ binary strings in the above example.
The above strings are arranged in ascending order by their values.
In our example, 00011 is the first string since it is the smallest,
00101 is the second string since it is bigger than 00011 and smaller than
any others.
Finally, 11000 is the tenth string since it is the largest.

Given $n$, $k$, and $m$, write a program to generate the $m$-th
$n$-bit binary string with exact $k$ 1's.

\bigskip\noindent{\bf The Input}

The input file may contain many instances.
Each instance consists of three numbers, $n$, $k$, and $m$.
You may assume that $n\le32$.

\bigskip\noindent{\bf The Output}

For each instance, print out the $m$-th $n$-bit binary string with
exactly $k$ 1's, and its corresponding decimal number.

\bigskip\noindent{\bf Sample Input}

\begin{verbatim}
7 3 10
16 8 10001
32 12 5000000
\end{verbatim}

\bigskip\noindent{\bf Sample Output}

\begin{verbatim}
n = 7, k = 3, m = 10
Decimal number = 28
Binary string = 0011100

n = 16, k = 8, m = 10001
Decimal number = 49897
Binary string = 1100001011101001

n = 32, k = 12, m = 5000000
Decimal number = 31475149
Binary string = 00000001111000000100010111001101
\end{verbatim}
\vspace{1cm }

\begin{center}
{\Large\bf Probelm C: Find the Rank}
\end{center}

\bigskip\noindent{\bf The Problem}

Given two sorted sequences of integers $A=a_1,a_2,\dots,a_m$ and
$B=b_1,b_2,b_n$.
Both sequences are sorted in nondecreasing order.
Let $s=\left\lceil{m \over 2}\right\rceil$ and
$t=\left\lceil{n \over 2}\right\rceil$.
Obviously, $x=a_s+b_t$ is one of the $nm$ elements $c_{i,j}=a_i+b_j$, where
$1 \le i \le m, 1 \le j \le n$.
Write a program to find the rank of $x$ when all these $nm$
elements are sorted in nondecreasing order.
In case that there are more than one element which have the same value
as $x$, let the rank of $x$ be the first of these
elements.

\bigskip\noindent{\bf The Input}

The input file may contain more than one instance.
Each instance consists of two sorted lists.
The first number in each list is the number of elements in that list.
This is then followed by the elements of that lists, sorted in
nondecreasing order.
You may assume that the maximum number of elements in each list is
no more than 2000.

\bigskip\noindent{\bf Output}

For each instance, your program must print the numbers $m$ and $n$,
together with the rank of $a_s+b_t$ in a format as shown in the Sample
Output.

\bigskip\noindent{\bf Sample Input}
\begin{verbatim}
5
15 20 34 36 71
4
18 25 50 100
8
100 101 105 151 151 180 200 240
5
80 88 100 200 202
\end{verbatim}

\noindent{\bf Sample Output}
\begin{verbatim}
m=5, n=4, rank of 59 is 7
m=8, n=5, rank of 251 is 14
\end{verbatim}
\vspace{1cm }

\begin{center}
{\Large\bf Problem D: Minimum Connection Cost}
\end{center}

\bigskip\noindent{\bf The Problem}

Let $V$ be a set of points $v_{1},v_{2},\ldots,v_{n}$ in the plane,
where $v_{i}=(x_{i},y_{i})$, $1\le i \le n$.
Assume that all the $x_{i}$'s and the $y_{i}$'s are integers less than
or equal to $255$.
Two points $v_i$ and $v_j$ are {\it connected} if either there is a
line between them, or there is another point $v_k$ such that both the
pair of points $v_i$ and $v_k$ and the pair of points $v_k$ and $v_j$
are connected.
Given a set of points, we want to construct a set of lines to make
every pair of points connected.
It is easy to see that $n-1$ lines are necessary and sufficient to
connect $n$ points.
Assume that the cost of the line connecting any pair of points $v_{i}$
and $v_{j}$ is $(x_{i}-x_{j})^2+(y_i-y_j)^2$.

For example, let the set of points be
$\{(100,100),(20,10),(20,20),(10,20)\}$.
The four lines $(100,100)$ to $(20,20)$, $(20,10)$ to $(20,20)$, and
$(20,20)$ to $(10,20)$ connects all pair of the given points.
The cost of the connection is $12800+100+100=13000$.

Write a program to compute the cost of connecting a given set of
points.

\bigskip\noindent{\bf The Input}

The input file contain more than one instance.
Each instance is a set of points in the plan.
The first number of an instance is $n$, the number of points in the
plan to be connected.
This is then followed by $n$ points.
Each point consists of two numbers, $x$ and $y$.

\bigskip\noindent{\bf The Output}

For each instance, print the minimum cost to connect the set of
points.
Note that you need not print the lines that connect these points.

\bigskip\noindent{\bf Sample Input}

\begin{verbatim}
2
0 1
1 0
4
100 100
20 10
20 20
10 20
\end{verbatim}

\bigskip\noindent{\bf Sample Output}
\begin{verbatim}
The minimum connection cost is 2.
The minimum connection cost is 13000.
\end{verbatim}
\vspace{1cm }

\begin{center}
{\Large\bf Problem E: Number of Combinations}
\end{center}

\bigskip\noindent{\bf The Problem}

The number of combinations of $n$ distinct items taken $m$ at a time is
given by the following formula:  $${n\choose m} = {{n!}\over{m!(n-m)!}}$$
for integers $m$ and $n$, with $0\le m\le n$.
Given two numbers $n$ and $m$, write a program to compute the value of
$n\choose m$.

\bigskip\noindent{\bf The Input}

The input file may contain more than one instance.
Each instance consists of two integers $m$ and $n$ with $0<m<n$.

\bigskip\noindent{\bf The Output}

The values of $n$, $m$, and $n\choose m$ must be printed in a format as
shown in the Sample Output.
Note that if the number of digits of $n\choose m$ is more than 50, it
should be warped to the next line.

\bigskip\noindent{\bf Sample Input}
\begin{verbatim}
5 2
3 3
123 4
\end{verbatim}

\bigskip\noindent{\bf Sample Output}
\begin{verbatim}
C(5,2)=10
C(3,3)=1
C(123,4)=9078630
\end{verbatim}
\vspace{1cm }

\begin{center}
{\Large\bf Problem F: Euler Tour}
\end{center}

\bigskip\noindent{\bf The Problem}

Let $G$ be a simple undirected graph with $n$ vertices $1,2,\dots,n$.
The graph $G$ can be represented by an {\it adjacency matrix} $A$.
Each element, $a_{i,j}$, of $A$ is either $0$ or $1$, and $(a_{i,j})=1$
if, and only if, there is an edge between vertex $i$ and vertex $j$.
For example, the following picture shows a graph at the left and its
adjacency matrix at the right.
\begin{verbatim}
1----2
 \  /                 0 1 1 0 0
  \/                  1 0 1 0 0
  3                   1 1 0 1 1 
  /\                  0 0 1 0 1
 /  \                 0 0 1 1 0
4----5
\end{verbatim}
A tour of $G$ is a sequence $v_0,e_1,v_1,e_2,v_2\dots,e_k,v_k$,
Such that $e_i$ is an edge that connects $v_{i-1}$ and $v_i$.
An Euler tour is a tour $v_0,e_1,v_1,e_2,v_2\dots,e_k,v_k$
such that $v_0 = v_k$, and each edge appears exactly once in the
tour.
Since the graph is simple, we will omit all the edges in the tour.
For example, the sequence 2,3,5,4,3,1,2 is an Euler tour of the above graph.
In this problem, we want to write a program to construct an Euler tour
for simple undirected graphs.
Note that not every graph has an Euler tour, and
if a graph has an Euler tour, the tour may not be unique.

\bigskip\noindent{\bf The Input}

The input file may contain more than one instance.
The first line of each instance is the number of vertices $n$.
It is then followed by $a_{i,j}$'s, arranged in the
order $a_{1,j}$, $j=1,2\dots,n$, $a_{2,j}$, $j=1,2\dots,n$, and so on.
You may assume that $n<100$.

\bigskip\noindent {\bf The Output}

If the graph has an Euler tour, the output is a sequence of vertices
which is an Euler tour of the graph.
If the graph has no Euler tour, print the string
``The graph is not Eulerian.''

\bigskip\noindent {\bf Sample Input}

\begin{verbatim}
5
0 1 1 0 0
1 0 1 0 0
1 1 0 1 1
0 0 1 0 1
0 0 1 1 0
3
0 1 0
1 0 1
0 1 0
\end{verbatim}

\bigskip\noindent {\bf Sample Output}
\begin{verbatim}
2 3 5 4 3 1 2
The Graph is not Eulerian.
\end{verbatim}

\end{document}
\bye

\endinput
